%%%%%%%%%%%%%%%%%%%%% chapter.tex %%%%%%%%%%%%%%%%%%%%%%%%%%%%%%%%%
%
% sample chapter
%
% Use this file as a template for your own input.
%
%%%%%%%%%%%%%%%%%%%%%%%% Springer-Verlag %%%%%%%%%%%%%%%%%%%%%%%%%%
%\motto{Use the template \emph{chapter.tex} to style the various elements of your chapter content.}
\chapter{A Survey on Related Work}
\label{survey} % Always give a unique label
% use \chaptermark{}
% to alter or adjust the chapter heading in the running head


%TODO: Abstract
\abstract{Coexistence between U-LTE and Wi-Fi networks is a deciding factor on the acceptance of U-LTE. As a result, a large number of studies have been carried out to identify what could be the affects that U-LTE may cause to Wi-Fi and which mechanisms could be used for ensure that these two technologies share the $5$ GHz unlicensed frequency band in an efficient and fair manner. This chapter presents a survey on related work to answer the following questions: (i) what issues arise from simultaneous operation of LTE and Wi-Fi in the same spectrum bands, (ii) which technology is affected the most, and (iii) which factors determine the impacts of U-LTE to Wi-Fi. It also identifies the strengths and weaknesses of existing solutions and suggests potential strategies to improve performance of these two technologies.}


\section{The Impacts of U-LTE on Wi-Fi Operation and Performance}

In \cite{original-LTE-Wi-Fi-VTC-2013}, extensive simulations have been performed to assess the performance of LTE and Wi-Fi coexisting in an office environment. Single-floor and multi-floor office environments with different assumptions on the density of Wi-Fi and LTE nodes have been considered. The simulation results in \cite{original-LTE-Wi-Fi-VTC-2013} have shown that, in the absence of any modification to LTE channel access mechanism, channel sharing between LTE and Wi-Fi networks is significantly unfair for Wi-Fi networks. While LTE only marginally loses (about $4$\% of the performance) when Wi-Fi is present on the same band, Wi-Fi could lose up to $70$\% performance in a sparse deployment ($1$ AP per system per floor) and to almost $100$\% in a dense deployment ($5$ APs per system per floor). Detailed investigations in \cite{original-LTE-Wi-Fi-VTC-2013} have indicated that Wi-Fi channel is blocked when LTE interference is present, and thus Wi-Fi nodes keep staying on the ``listen'' mode most of the time.

The authors in \cite{original-LTE-Wi-Fi-WCNC-2013} present observations similar to those in \cite{original-LTE-Wi-Fi-VTC-2013} on the effects of unmodified LTE to Wi-Fi networks in the shared frequency band. Specifically, when network load is increased, LTE performance suffers only a minor degradation, while Wi-Fi performance drops significantly. This can be explained by the increasing LTE occupancy on the shared band. LTE does not follow the same rules as Wi-Fi in shared medium access. When there is ongoing transmission on the channel, while Wi-Fi politely defers its transmission, LTE always choose to transmit by selecting a more robust transmission mode by adapting its modulation and channel coding scheme in order to cope with the higher interference. This aggressive behavior quickly results in a situation where LTE terminals take all transmission opportunities while Wi-Fi devices are locked in back-off procedures. Unfortunately, the results in \cite{original-LTE-Wi-Fi-WCNC-2013} have also demonstrated that the severity of this negative impact on Wi-Fi can be efficiently controlled by restricting LTE activity.

The authors in \cite{LTE-U-PIMRC-2014} analyze the performance degradation of Wi-Fi in the presence of LTE-U. The probability of Wi-Fi accessing the channel is used as the main metric. Numerical results in \cite{LTE-U-PIMRC-2014} indicate that Wi-Fi is negatively affected by conventional LTE operation due to LTE's almost continuous transmission that subsequently blocks Wi-Fi. Specifically, given two modes of operations currently proposed for LTE-U in the unlicensed spectrum, the ``off'' period presented by the LTE protocol is too short for Wi-Fi users to access to the channel. As a result, Wi-Fi is at risk of spending a significant amount of time in the ``listening'' mode when LTE transmission is present in the same channel.

The work in \cite{U-LTE-Google-WP} presents initial investigations on the coexistence of two versions of license-anchored U-LTE (i.e., LTE-U and LAA-LTE) and Wi-Fi in $5$ GHz frequency band. Results in \cite{U-LTE-Google-WP} show that LTE-U poorly coexists with Wi-Fi primarily due to two factors: (i) the incompatibility of LTE-U's duty-cycling mechanism with Wi-Fi equipment and (ii) the lack of an effective coexistence mechanism in scenarios where LTE-U and Wi-Fi devices hear each other at moderate but non-negligible power levels. Additionally, LAA-LTE with LBT does not by itself guarantee successful coexistence with Wi-Fi and other purely unlicensed technologies. The results in \cite{U-LTE-Google-WP} were submitted to FCC in June 2015 to demonstrate that, although any wireless technology should have the ability to utilize unlicensed spectrum within the FCC's rules, U-LTE has the potential to crowd out unlicensed services.

An experiment-based study on the effect of LTE-U to Wi-Fi is presented in \cite{LTE-U-CableLabs}. The LTE signal level is set higher than the Wi-Fi clients' LBT energy detection threshold (i.e., when LTE is on, the Wi-Fi client should sense their presence and not transmit). Wi-Fi throughput and latency are measured when data is transmitted through the Wi-Fi network with varying duty cycles and periods of LTE signals. The results in \cite{LTE-U-CableLabs} indicate that, as expected, increasing the LTE-U duty cycle degrades both Wi-Fi throughput and latency performance since it decreases Wi-Fi transmission opportunity accordingly. If the duty cycle period is too high, Wi-Fi latency is negatively impacted (while Wi-Fi throughput is nearly unchanged, given the same duty cycle) since Wi-Fi frames have to be buffered during long LTE ``on'' period. However, if the duty cycle period is configured as too low (e.g., $10$ msec), Wi-Fi throughput degrades due to the fact that LTE ``on'' and ``off'' periods are too short for Wi-Fi users to access to the channel and to complete their transmissions, respectively. Furthermore, the authors in \cite{LTE-U-CableLabs} indicate that LTE-U duty cycle cannot strictly results in corresponding air time and throughput sharings. For example, with a duty cycle of $50$\%, LTE-U is likely to capture more than $50$\% of the channel resources. The reason is that when LTE-U starts its transmissions (regardless of ongoing Wi-Fi frame transmissions), many Wi-Fi frames are corrupted. Transmission failures lead to multiple frame re-transmissions and, more importantly, mistakenly force Wi-Fi transceivers to operate at lower rates (in this case, lowering the channel coding and modulation modes is not necessary and waste of channel efficiency).


\section{Existing Solutions to Address the U-LTE and Wi-Fi Coexistence Concern}

Various coexistence mechanisms proposed for U-LTE are surveyed in \cite{U-LTE-survey-2014, U-LTE-survey-2015, U-LTE-5G-2015}. They include: Dynamic Channel Selection (DCS), power control, opportunistic secondary cell ``off'', CSAT (in LTE-U), and LBT (in LAA-LTE). When U-LTE and Wi-Fi share the common $5$ GHz radio frequency band, those mechanisms are found to useful to reduce RFI and improve the spectrum utilization efficiency. The roles of each of them, however, greatly varies depending on network/system parameters including network scale, node density, deployment/radio environment (i.e., indoor, outdoor, short range, long range, ...), network load profiles, etc. 

In order to see how LBT mechanisms employed by LAA-LTE can help for the coexistence, a simulation-based study is carried out and reported in \cite{LBT-CableLabs-2014}. LBE LBT specified by ETSI \cite{LBT-ETSI-2014} and IEEE 802.11e Enhanced Distributed Channel Access (EDCA) are assumed for LAA-LTE and Wi-Fi, respectively. The most important observation from \cite{LBT-CableLabs-2014} is that LBT compliant to ETSI regulation is not sufficient for fair coexistence: Wi-Fi STAs have much lower probability of successful channel access compared to LAA-LTE users. One major reason for this phenomenon is the non-exponential back-off LBT employed by LAA-LTE. Unfortunately, no form of exponential back-off LBT is studied in \cite{LBT-CableLabs-2014}.

In \cite{U-LTE-Wi-Fi-Qualcomm-2014, LTE-U-Wi-Fi-LTE-U-Forum-2015, LTE-U-Qualcomm-2015, U-LTE-Wi-Fi-Qualcomm-FCC-2015}, the performance of LTE-U and LAA-LTE and Wi-Fi in a shared frequency band is evaluated. DCS and opportunistic secondary cell ``off'' in unlicensed spectrum (U-LTE small cells would release the unlicensed carriers and fall back to the anchor carrier in licensed spectrum at low traffic load) are jointly used with CSAT and LBT. The results show that co-existence has a negative but controllable impact on Wi-Fi performance. In \cite{U-LTE-Wi-Fi-Qualcomm-2014, LTE-U-Wi-Fi-LTE-U-Forum-2015, LTE-U-Qualcomm-2015}, LTE-U can be a better neighbor to Wi-Fi than Wi-Fi to itself in some scenarios. The underlying design that allows LTE-U to achieve high spectral efficiency while being a good neighbor to Wi-Fi is achieved through a set of carefully designed coexistence techniques, including DCS, secondary cell ``duty cycle'' in unlicensed spectrum (i.e., CSAT), and opportunistic secondary cell ``off'' in unlicensed spectrum. Specifically, in scenarios where the density of Wi-Fi APs and small cells is low or moderate, DCS and opportunistic secondary cell ``off'' are sufficient to meet the coexistence requirement. When LTE-U devices replace Wi-Fi devices, they can achieve significantly higher throughputs due to their high spectral efficiency. In addition, the performance of neighboring Wi-Fi is unchanged or even slightly improved since LTE-U devices can finish transmission faster and incur less interference. However, as the density of Wi-Fi devices and LTE-U small cells is high, DCS and opportunistic secondary cell ``off'' alone cannot guarantee harmonious coexistence with Wi-Fi and therefore CSAT or LBT is required. Resutls in \cite{LTE-U-Wi-Fi-LTE-U-Forum-2015, U-LTE-Wi-Fi-Qualcomm-FCC-2015} were submitted to FCC in 2015 to support U-LTE technologies.

A systematic and large-scale network-wide study of LAA-LTE and Wi-Fi performance in a wide range of realistic deployment scenarios and network densities in the unlicensed $5$ GHz band is presented in \cite{LTE-U-ICC-WS-2015}. The simulation results in all considered coexistence scenarios demonstrate that both LAA-LTE and Wi-Fi significantly benefit from the large number of available channels and the isolation provided by building shielding at $5$ GHz. They also suggest that deploying LAA-LTE with a random channel selection scheme is feasible for lower network densities. For typical indoor deployments of high density, implementing LTE-U interference-aware channel selection with respect to Wi-Fi is superior to LBT in terms of achieved throughput for both technologies. Additionally, LBT can increase LAA-LTE user throughput when multiple outdoor LAA-LTE networks deployed by different cellular operators coexist.

The work in \cite{Enhanced-LTE-U-thesis-2015} investigates the behavior and performance of two existing LBT mechanisms that are designed following the coexistence standard specified by ETSI \cite{LBT-ETSI-2014}: LBE and FBE-based mechanisms. The Jain's fairness index has been used to access the coexistence of LAA-LTE using these two LBT mechanisms and Wi-Fi using CSMA-CA. The simulations in \cite{Enhanced-LTE-U-thesis-2015} show that FBE-based mechanism using fixed contention window penalizes the channel access opportunity of Wi-Fi's CSMA-CA using adaptive contention window. They also reveal that FBE-based mechanism tends to aggressively occupy the channel. In some cases, Wi-Fi is starved with very less (or even no) chance on the channel access. This poor fairness is mainly caused by the short CCA sensing period of FBE-based mechanism. CCA is applied only once and then FBE-based mechanism may start its transmission immediately while LBE and Wi-Fi-based mechanisms are still decrementing their respective back-off counters. The fairness is worsened with longer FBE's frames. Another observation is that, again due to equal CCA sensing time, when multiple FBE-based equipment are contending for the channel, they are prone to serious collisions (if they are accidentally synchronized) or suffer a significant unfairness (if they are asynchronous). To cope with those issues, tuning the values of back-off scaler ($q$) to extend the contention window size and using CCA procedure similar to that of LBE-based mechanism have been suggested for LBE and FBT-based mechanism, respectively. The results in \cite{Enhanced-LTE-U-thesis-2015} demonstrate that the modified LBE-based mechanism still cannot sufficiently improve the fairness with others. This could be because simply empirically tuning back-off scaler while keeping the CCA principle unchanged cannot compensate for exponential growth of window size adopted by Wi-Fi's CSMA-CA. The modified FBE-based mechanism can offer better fairness when coexisting with Wi-Fi.

A comparison of LTE-U and LAA-LTE is presented in \cite{LBT-CSAT-2015}. The analysis in \cite{LBT-CSAT-2015} shows that for sufficiently long LTE transmission times, the LTE throughputs achieved by CSAT and LBE are almost identical. However, for shorter LTE transmission times, LTE-U provides lower LTE throughput than LAA-LTE due to higher LTE/Wi-Fi collision probability of LTE-U. Besides, while shorter LTE transmission time decreases the tail of the Wi-Fi delay distribution, the percentage of packets that suffer from long delays increases. The results also indicate that when appropriately configured, LTE-U and LAA-LTE provide the same level of fairness to Wi-Fi. The selection of co-existence mechanisms is primarily driven by the operator's interests that include implementation complexity, LTE throughput, operational and management costs as well as strategic decisions on targeted markets.

Coordinated coexistence between U-LTE and Wi-Fi is investigated in \cite{U-LTE-5G-2015, Coordinated-LTE-U-Wi-Fi-2015}. The authors in \cite{U-LTE-5G-2015} propose a method of centralized system management to combine LTE-U and Wi-Fi through network function virtualization (NFV) interconnections. It may enable seamless transfer of resources between LTE-U and Wi-Fi using in-the-cloud control of distributed access points. \textit{However}, only \textit{conceptual} network architectures and mechanisms are presented \cite{U-LTE-5G-2015}. The authors in \cite{Coordinated-LTE-U-Wi-Fi-2015} present a Software Defined Networking (SDN) architecture to support logically-centralized dynamic spectrum management involving multiple autonomous networks to improve spectrum utilization and facilitate co-existence. The basic design goal is to support the seamless communication and information dissemination required for coordination of heterogeneous networks. The system consists of two-tiered controllers are mainly responsible for the control plane. Global Controller (GC) acquires and processes global network state information (radio coverage maps, coordination algorithms, policy and network evaluation matrices, etc.) and controls the flow of information between RCs and databases based on authentication and other regulatory policies. Regional Controllers (RCs) acquire local visibility needed for radio resource allocation at wireless devices: device location, frequency band, duty cycle, power level, and data rate, etc. Joint power control and time division channel access optimizations are proposed. Analytical results in \cite{Coordinated-LTE-U-Wi-Fi-2015} demonstrate that, \textit{with full buffer traffic assumption}, centralized optimization approaches can provide fair access to the spectrumfor LTE-U and Wi-Fi networks.

An experimental evaluation of U-LTE interference effects on Wi-Fi performance under various network conditions along with some suggestions for better coexistence of U-LTE and Wi-Fi networks are presented in \cite{LTE-U-Experiment-ICC-WS-2015}. Various system parameters (bandwidth, center frequency, etc.) are swept to identify the most significant ones that determine the levels of LTE interference introduced to Wi-Fi carrier sense and performance. The results indicate that Wi-Fi throughput can be heavily degraded by LAA-LTE transmissions with $3$/$5$/$10$ MHz bandwidth (especially $3$/$5$ MHz). Besides, LAA-LTE transmissions can have small impact on Wi-Fi throughput when using a $1.4$ MHz channel with center frequencies located on the guard bands or the center frequencies of Wi-Fi channels. However, the authors in \cite{LTE-U-Experiment-ICC-WS-2015} do not clearly define what LAA-LTE really mean in their work. It seems to be that they simply perform experiments with conventional LTE transceivers of varying power spectral densities and do not incorporate any coexistence mechanism into the LTE system.



\begin{thebibliography}{99.}%
\bibitem{U-LTE-survey-2014}F.~Abinader, E.~Almeida, F.~Chaves, A.~Cavalcante, R.~Vieira, R.~Paiva, A.~Sobrinho, S.~Choudhury, E.~Tuomaala, K.~Doppler, and V.~Sousa, ``Enabling the coexistence of {LTE} and {Wi-Fi} in unlicensed bands,'' \emph{IEEE	Communications Magazine}, vol.~52, no.~11, pp. 54--61, Nov 2014.	
	
\bibitem{U-LTE-survey-2015} H.~Zhang, X.~Chu, W.~Guo, and S.~Wang, ``Coexistence of {Wi-Fi} and heterogeneous small cell networks sharing unlicensed spectrum,'' \emph{IEEE	Communications Magazine}, vol.~53, no.~3, pp. 158--164, March 2015.

\bibitem{U-LTE-5G-2015}A.~Al-Dulaimi, S.~Al-Rubaye, Q.~Ni, and E.~Sousa, ``{5G} communications race: Pursuit of more capacity triggers {LTE} in unlicensed band,'' \emph{IEEE Vehicular Technology Magazine}, vol.~10, no.~1, pp. 43--51, March 2015.

\bibitem{original-LTE-Wi-Fi-VTC-2013}A.~M. Cavalcante \emph{et~al.}, ``Performance evaluation of {LTE} and {Wi-Fi} coexistence in unlicensed bands,'' in \emph{Proceedings of 2013 IEEE	Vehicular Technology Conference (VTC Spring)}, 2013, pp. 1--6.

\bibitem{original-LTE-Wi-Fi-WCNC-2013} T.~Nihtila, V.~Tykhomyrov, O.~Alanen, M.~Uusitalo, A.~Sorri, M.~Moisio, S.~Iraji, R.~Ratasuk, and N.~Mangalvedhe, ``System performance of {LTE} and
{IEEE} 802.11 coexisting on a shared frequency band,'' in \emph{2013 IEEE Wireless Communications and Networking Conference (WCNC)}, April 2013, pp. 1038--1043.

\bibitem{LTE-U-PIMRC-2014} A.~Babaei, J.~Andreoli-Fang, and B.~Hamzeh, ``On the impact of {LTE-U} on {Wi-Fi} performance,'' in \emph{2014 IEEE 25th Annual International Symposium	on Personal, Indoor, and Mobile Radio Communication (PIMRC)}, Sept 2014, pp.1621--1625.

\bibitem{U-LTE-Google-WP} N.~Jindal and D.~Breslin, ``{LTE} and {Wi-Fi} in unlicensed spectrum: A coexistence study,'' white paper, Google.

\bibitem{LTE-U-CableLabs} J.~Padden, ``{Wi-Fi} vs. duty cycled {LTE}: A balancing act,'' CableLabs. [Online]. Available: \url{http://www.cablelabs.com/wi-fi-vs-duty-cycled-lte/}

\bibitem{LBT-CableLabs-2014} J.~P. A.~Babaei and J.~Andreoli-fang, ``Overview of {EU} {LBT} and its effectiveness for coexistence of {LAA LTE} and {Wi-Fi},'' IEEE 802.19-14/0082r0, CableLabs, Nov. 2014.

\bibitem{U-LTE-Wi-Fi-Qualcomm-2014} ``{LTE} in unlicensed spectrum: Harmonious coexistence with {Wi-Fi},'' whitepaper, Qualcomm Inc., Jun. 2014.

\bibitem{LTE-U-Wi-Fi-LTE-U-Forum-2015} ``{LTE-U} technical report: Coexistence study for {LTE-U} {SDLV1.0(2015-02)},'' LTE-U Forum, Tech. Rep., 2015.

\bibitem{LTE-U-Qualcomm-2015}A.~Sadek, T.~Kadous, K.~Tang, H.~Lee, and M.~Fan, ``Extending {LTE} to unlicensed band - merit and coexistence,'' in \emph{Communication Workshop (ICCW), 2015 IEEE International Conference on}, June 2015, pp. 2344--2349.

\bibitem{U-LTE-Wi-Fi-Qualcomm-FCC-2015}``Office of engineering and technology and wireless telecommunications bureau seek information on current trends in {LTE-U} and {LAA} technology,''
Comments of Qualcomm Incoporated, Qualcomm Inc., Jun. 2015.

\bibitem{LTE-U-ICC-WS-2015} A.~Voicu, L.~Simic, and M.~Petrova, ``Coexistence of pico- and femto-cellular {LTE}-unlicensed with legacy indoor {Wi-Fi} deployments,'' in \emph{2015 IEEE
	International Conference on Communication Workshop (ICCW)}, June 2015, pp.2294--2300.

\bibitem{Enhanced-LTE-U-thesis-2015}A.~Kanyeshuli, ``{LTE} in unlicensed band: Medium access and performance evaluation,'' Master's thesis, University of Agder, Norway, May 2015.

\bibitem{LBT-CSAT-2015}C.~Cano and D.~J. Leith, ``Unlicensed {LTE}/{WiFi} coexistence: Is {LBT} inherently fairer than {CSAT}?'' \emph{Scentific Publication Data: Computer
	Science - Networking and Internet Architecture}, nov 2015.

\bibitem{Coordinated-LTE-U-Wi-Fi-2015}S.~Sagari, S.~Baysting, D.~Saha, I.~Seskar, W.~Trappe, and D.~Raychaudhuri,``Coordinated dynamic spectrum management of {LTE-U} and {Wi-Fi} networks,''in \emph{2015 IEEE International Symposium on Dynamic Spectrum Access Networks (DySPAN)}, Sept 2015, pp. 209--220.

\bibitem{LTE-U-Experiment-ICC-WS-2015}Y.~Jian, C.-F. Shih, B.~Krishnaswamy, and R.~Sivakumar, ``Coexistence of {Wi-Fi} and {LAA-LTE}: Experimental evaluation, analysis and insights,'' in
\emph{2015 IEEE International Conference on Communication Workshop (ICCW)}, June 2015, pp. 2325--2331.
\end{thebibliography}
