%%%%%%%%%%%%%%%%%%%%% chapter.tex %%%%%%%%%%%%%%%%%%%%%%%%%%%%%%%%%
%
% sample chapter
%
% Use this file as a template for your own input.
%
%%%%%%%%%%%%%%%%%%%%%%%% Springer-Verlag %%%%%%%%%%%%%%%%%%%%%%%%%%
%\motto{Use the template \emph{chapter.tex} to style the various elements of your chapter content.}


\chapter{Network-aware Adaptive Listen Before Talk Co-existence Mechanism}
\label{intro-NALT} % Always give a unique label
% use \chaptermark{}
% to alter or adjust the chapter heading in the running head

%\abstract*{Each chapter should be preceded by an abstract (10--15 lines long) that summarizes the content. The abstract will appear \textit{online} at \url{www.SpringerLink.com} and be available with unrestricted access. This allows unregistered users to read the abstract as a teaser for the complete chapter. As a general rule the abstracts will not appear in the printed version of your book unless it is the style of your particular book or that of the series to which your book belongs.
%Please use the 'starred' version of the new Springer \texttt{abstract} command for typesetting the text of the online abstracts (cf. source file of this chapter template \texttt{abstract}) and include them with the source files of your manuscript. Use the plain \texttt{abstract} command if the abstract is also to appear in the printed version of the book.}

\abstract{In the absence of coordination between radio access technologies, and with the goal of deploying unlicensed LTE without requiring changes to the Wi-Fi MAC layer, it falls to the LTE base stations to ensure fair coexistence.   As we have seen, Wi-Fi employs a fairly simple multiple access method which can easily lead to Wi-Fi stations being barred from the channel if LAA-LTE is not designed to promote fairness.  If no changes are to be made to Wi-Fi devices, then the greatest gains in fair coexistence are achieved when unlicensed LTE behaves in as Wi-Fi like a manner as possible, however, this may not allow LTE to make the best use of the channel.  In this chapter, a network-aware adaptive LBT mechanism (NALT) is proposed which monitors both channel conditions and usage activity to maximize its transmission opportunities, while maintaining fair sharing of the channel, in a way that is transparent to incumbent Wi-Fi devices.}

\section{Background and Theoretical Basis}
\label{background}
As discussed in Chapter \ref{overview-wifi}, \mbox{Wi-Fi} employs a fairly simple multiple access strategy which can be easily overwhelmed if competing devices are not designed for fair coexistence. The \mbox{Wi-Fi}  MAC protocol is based on a probabilistic model of channel access which minimizes collisions through the use of random exponential backoff \cite{80211}.  While the ETSI LBT mechanism on which the recommended LBT mechanism for \mbox{LAA-LTE} is based is also probabilistic, it employs a random backoff from a fixed set of possible backoff values \cite{3gpp}, which does not attempt to reduce the probability of collision on repeated failed transmission.  Thus, if a collision occurs, \mbox{Wi-Fi} will react by reducing its probability of gaining access to the channel, relative to a device modelled on the ETIS LBT mechanism.  Additionally, \mbox{LAA-LTE} used for supplemental downlink or carrier aggregation is expected to align subframes with the licensed band, and such subframes have a duration of $1$ms, which is significantly longer than the .  Combined, these two factors will lead to \mbox{LAA-LTE} transmissions occupying the channel for significantly longer than an average competing \mbox{Wi-Fi} station would resulting in unfair sharing of the channel even if the number of channel accesses is equal.  Since \mbox{Wi-Fi} stations may operate at any of several modulation and coding schemes, it is also difficult to provide throughput fairness  unless \mbox{LAA-LTE} base stations monitor both however, using the principles developed for the $802.11$e Enhanced Distributed Channel Access (EDCA) function for service differentiation between traffic priorities, there are mechanism to provide airtime fairness \cite{80211}.  

In EDCA, \mbox{Wi-Fi} parameters such as contention window and inter-frame spacing are set up to provide different quality of service to different types of traffic \cite{80211}.  By changing these parameters, it is possible to impact the probability of channel access in a predictable way.  Specifically, the relationship between minimum contention window size for two traffic classes, and their relative proportion of channel access was found to be 

\begin{equation}\label{cw}
\frac{\theta_i}{\theta_j} \approx \frac{CW^j_{min}}{CW^i_{min}}
\end{equation}
where, $\frac{\theta_i}{\theta_j}$ is the proportional airtime class $i$ sees relative to class $j$, and $CW^j_{min}$ is the minimum contention window used by class $j$ \cite{yoon,chou}.  Using this relationship, greater airtime fairness can be achieved by tuning the $CW_{min}$ values used by competing stations. Further, it is desired to avoid any changes to \mbox{Wi-Fi} MAC layers, and fairer coexistence can be achieved by designing a more \mbox{``\mbox{Wi-Fi}-like"} MAC layer for \mbox{LAA-LTE} \cite{kwan}.

To balance airtime between \mbox{LAA-LTE} and \mbox{Wi-Fi}, it is necessary to treat all \mbox{Wi-Fi} stations as a single traffic class, which is allocated a proportion of the airtime depending on both the number of members of the class and the average duration a channel access.  For example, if a \mbox{Wi-Fi} channel access takes half the time of a \mbox{LAA-LTE} channel access, in the case of a single \mbox{LAA-LTE} station competing with a single \mbox{Wi-Fi} station, the \mbox{Wi-Fi} station should receive twice as many transmission opportunities as the \mbox{LAA-LTE} station in order to achieve equal airtime.  If there were two \mbox{Wi-Fi} stations, in order for each to have equal airtime, the \mbox{LAA-LTE} station should receive one quarter as many transmission opportunities as the combined \mbox{Wi-Fi} stations, so that proportionally each of the three stations would receive equal airtime on average.  Substituting values in Eq. \ref{cw} and solving for the required $CW_{min}$ values to realize this relation yields,

\begin{equation}\label{cwlte}
CW^{LTE}_{min} = \rho\cdot n_{WiFi} \cdot{CW^{WiFi}_{min}}
\end{equation}
where $\rho$ is the ratio of \mbox{LAA-LTE} transmission time to average \mbox{Wi-Fi} transmission time and $n_{WiFi}$ is the number of competing \mbox{Wi-Fi} stations.  This relation provides an approximation of the optimal $CW^{LTE}_{min}$ to provide airtime fairness, however, the \mbox{Wi-Fi} traffic class may be made up of stations which are using different transmission rates and $CW_{min}$ values.  In order to estimate the $CW^{WiFi}_{min}$ to use in Eq. \ref{cwlte}, and adjust to changing network topologies, an estimate of the average current $CW^{WiFi}$ can be obtained by an estimate of the number of collisions which are being observed on the channel.  In a strictly \mbox{Wi-Fi} system, the probability of collision in a saturated network is given by, 

\begin{equation}\label{pcollision}
p = 1-(1-\sfrac{1}{CW_{avg}})^{n-1}
\end{equation}
where $CW_{avg}$ is the average contention window currently being employed in the network, and $n$ is the number of competing stations \cite{vu}.  Rearranging and solving for $CW_{avg}$,
\begin{equation}\label{cwavg}
CW_{avg} = \frac{1}{1-e^{ \sfrac{ln(1-p)}{(n-1)} }}
\end{equation}
Eq. \ref{cwavg} provides the average contention window size for all stations.  In order to consider only the average contention window size for the \mbox{Wi-Fi} stations, and noting the optimal $CW^{LTE}_{min}$ to $CW^{WiFi}_{min}$ ratio, we can estimate $CW^{WiFi}_{min}$ as
\begin{equation}\label{cwwifi}
CW^{WiFi}_{avg} = CW_{avg}\left ( \frac{n_{WiFi} + n_{LTE}}{n_{WiFi} + \rho\cdot n_{LTE}} \right )
\end{equation}
then from Eq. \ref{cwlte}, we set 
\begin{equation}\label{cwlteopt}
CW^{LTE} = \rho \cdot n_{WiFi} \cdot{CW^{WiFi}_{avg}}
\end{equation}

To facilitate the use of Eq. \ref{cwlteopt} to ensure fair coexistence, it is assumed that the \mbox{LAA-LTE} base station is able to analyze the traffic on the channel and determine the number of competing \mbox{Wi-Fi} stations, as is assumed in other coexistence mechanisms.  It is further assumed the \mbox{LAA-LTE} base station is able to decode \mbox{Wi-Fi} headers to determine the modulation and coding schemes employed by the \mbox{Wi-Fi} stations to estimate transmission durations, that successfully decoding a \mbox{Wi-Fi} transmission implies that the transmission was successful since other \mbox{Wi-Fi} stations will also hear and not attempt to transmit (i.e. ignoring the hidden terminal problem), and that collisions involving \mbox{LAA-LTE} transmissions can be reported to the base station on control channels in the licensed spectrum.  Probability of collision, $p$, in Eq. \ref{cwavg} is estimated by the observed \mbox{LAA-LTE} collisions to the number of \mbox{LAA-LTE} channel uses.  Noting that these are empirical estimates of the true statistics, which may be highly inaccurate for few samples, $CW^{LTE}$ is also adjusted directly from the knowledge of channel activity so that the ratio of successful \mbox{Wi-Fi} to \mbox{LAA-LTE} transmissions approaches the desired airtime ratio. 


\section{Proposed Mechanism}
\label{proposed}
In order to make \mbox{LAA-LTE} behave more like \mbox{Wi-Fi}, the contention window used by \mbox{LAA-LTE} must increase as the number of collisions increases.  To facilitate fair airtime allocations across all competing devices, the contention window should follow Eq. \ref{cwlteopt}.  Based on the limitations of the estimates, and to ensure that the contention window stays within reasonable bounds, the maximum and minimum values for $CW^{LTE}$ are chosen to match the range of possible values for \mbox{Wi-Fi} \cite{80211}.  

Combining these requirements, and the preceding equations and assumptions, at each time instance an \mbox{LAA-LTE} station will estimate the average \mbox{Wi-Fi} contention window as follows:
\begin{align*}
CW^{WiFi}_{avg} = CW^{WiFi}_{min}, &\text{if \{\mbox{Wi-Fi} Tx\}$> \rho\cdot$\{\mbox{LAA-LTE} Tx\}}\\ 
&\text{Otherwise, update according to Eq. \ref{cwwifi}.}
\end{align*}
Thus, \mbox{LAA-LTE} will follow the same backoff procedure as \mbox{Wi-Fi}, and increase its contention window after a collision according to
\begin{align}
&CW^{LTE}=\;\;min\left[max\left(CW^{LTE}*2,\rho \cdot CW^{WiFi}_{avg}\right), CW^{LTE}_{MAX}\right]
\end{align}
and decreasing its contention window after a successful transmission according to 
\begin{align}
&CW^{LTE}=\;\;min\left[max\left(CW^{LTE}_{MIN},\rho \cdot CW^{WiFi}_{avg}\right), CW^{LTE}_{MAX}\right]
\end{align}


\section{Performance Evaluation}
To evaluate the performance of NALT, a high-level MATLAB simulation of a single \mbox{LAA-LTE} device contending with several \mbox{Wi-Fi} stations was developed and the proportion of channel accesses by each class of devices was tracked.  Simulating a single \mbox{LAA-LTE} device is reasonable since \mbox{LAA-LTE} user equipment can be coordinated via licensed band control channels, with scheduling done by the base station so that there is coordinated channel accesses for both uplink and downlink traffic.
\label{perf-eval}

\subsection{System Model}
\label{sys-model}
For simplicity, we assume that both \mbox{LAA-LTE} and \mbox{Wi-Fi} stations use the same modulation and coding scheme and channel bandwidth, resulting in a data rate of $135$ Mbps.  Other than the adaptive contention window, the channel occupancy, and minimum time idle were modeled after ETSI LBE LBT and the proposed mechanisms for \mbox{LAA-LTE} \cite{3gpp}. The other pertinent simulation parameters are listed in Table \ref{params}.
\begin{table}[!h]
	\centering
	\caption{NALT Simulation Parameters}
	\label{params}
	\begin{tabular}{l|l}
		Slot Duration                                                                      & 9 $\mu$s    \\ \hline
		DIFS                                                                               & 34 $\mu$s   \\ \hline
		SIFS                                                                               & 16 $\mu$s   \\ \hline
		\mbox{Wi-Fi} MPDU                                                                         & 1536 bytes  \\ \hline
		\begin{tabular}[c]{@{}l@{}}\mbox{Wi-Fi} Tx Duration\\ (Frame Tx + SIFS +ACK)\end{tabular} & 198 $\mu$s  \\ \hline
		\mbox{LAA-LTE} Tx Duration                                                                  & 1000 $\mu$s \\ \hline
		Number of competing \mbox{Wi-Fi} stations												& $1 - 10$ \\ \hline
		Simulated time duration																	   & 10 s
	\end{tabular}
\end{table}

\subsection{Simulation Results}
\label{sim-results}
Since the coexistence mechanism is probabilistic, $1000$ trials were run for each considered network topology of between $1$ and $10$ \mbox{Wi-Fi} stations contending with a \mbox{LAA-LTE} station.  The average number of successful channel accesses was tracked across all trials and is related to airtime by the transmission duration of each class of devices.  The resulting proportion of airtime for each device when using NALT is shown in Fig. \ref{results}.
\begin{figure}[t]
	\centering
	\subfloat[With LAA-LTE using NALT]{\includegraphics[width=3.5in]{figures/withadapt}%
		\label{results}}
	\hfil
	\subfloat[With LAA-LTE using ETSI LBE LBT]{\includegraphics[width=3.5in]{figures/withoutadapt}%
		\label{fig_second_case}}
	\caption{Airtime allocations for each station type, normalized to number of stations, with and without NALT.}
	\label{compresults}
\end{figure}
For comparison, the simulation was run with the same parameters as in Table \ref{params}, but with a fixed contention window size of $16$, corresponding to the approximate midpoint of possible values under ETSI LBE LBT \cite{3gpp}. The resulting airtime allocations per device are shown in Fig. \ref{compresults}.

\section{Discussion and Future Work}
\label{future-work}
NALT requires no changes to \mbox{Wi-Fi} devices and in high-level simulations it shows promise in providing fair coexistence. As noted, several assumptions were made which affect the results.  It is reasonable that \mbox{LAA-LTE} would be able to analyze the channel and determine the number or competing \mbox{Wi-Fi} stations as well as their transmission bit rates, from the \mbox{Wi-Fi} preamble and MAC header.  The simulation assumed all \mbox{Wi-Fi} stations were using the same data rate, which should provide the same results as an average data rate, but the impact on individual \mbox{Wi-Fi} stations utilizing the channel access opportunities in a multi-rate environment was not explored.  Further, the impacts of hidden terminals, non-saturated stations, and lossy channels, were not explored, and may have interesting implications. The assumption that up/downlink traffic within a single \mbox{LAA-LTE} network would be centrally coordinated by the base station simplifies the problem. However, NALT does not consider the scenario where several uncoordinated \mbox{LAA-LTE} networks operate in the same frequency band.  Without coordination between unrelated \mbox{LAA-LTE} networks, contention between \mbox{LAA-LTE} user equipment, whether or not \mbox{Wi-Fi} devices are also present, will present a further challenge. Treating \mbox{LAA-LTE} devices in the same way as \mbox{Wi-Fi} devices when estimating contention parameters and average transmission rates and durations would potentially render the outcome invalid.  Further simulations of this scenario, and possibly adjustments to the algorithm to account for different classes of devices, is a necessary next step in the development of NALT.

%%%%%%%%%%%%%%%%%%%%%%%% referenc.tex %%%%%%%%%%%%%%%%%%%%%%%%%%%%%%
% sample references
% %
% Use this file as a template for your own input.
%
%%%%%%%%%%%%%%%%%%%%%%%% Springer-Verlag %%%%%%%%%%%%%%%%%%%%%%%%%%

\begin{thebibliography}{99.}%
	\bibitem{3gpp}{3GPP}, ``Feasibility study on licensed-assisted access to unlicensed spectrum,'' \emph{{TR 36.889 v13.0.0}}, July 2015.
	\bibitem{chou} C.~T. Chou, K.~G. Shin, and S.~Shankar, ``{Contention-based Airtime Usage Control in Multirate IEEE 802.11 Wireless LANs},'' \emph{{IEEE/ACM Transactions on Networking}}, vol.~14, no.~6, pp. 1179--1192, December 2006.
	
	\bibitem{80211}``{IEEE Standard for information technology--Telecommunications and information exchange between systems Local and metropolitan area networks--Specific requirements Part 11: Wireless LAN Medium Access Control (MAC) and Physical Layer (PHY) Specifications},'' \emph{{IEEE Std 802.11-2012 (Revision of IEEE Std 802.11-2007)}}, pp. 818--972, March 2012.
	
	
	

	
	\bibitem{vu} H.~L. Vu and T.~Sakurai, ``{Collision probability in saturated IEEE 802.11 networks},'' in \emph{{Australian Telecommunication Networks \& Applications Conference}}, September 2006, pp. 1--5.
	\bibitem{yoon} {J. Yoon, S. Yun, et al}, ``{Maximizing Differentiated Throughput in IEEE 802.11e Wireless LANs},'' in \emph{{ Proceedings of the 31st IEEE Conference on Local Computer Networks}}, vol.~14, no.~6, November 2006, pp. 411--417.
\end{thebibliography}


