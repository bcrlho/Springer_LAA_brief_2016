%%%%%%%%%%%%%%%%%%%%% chapter.tex %%%%%%%%%%%%%%%%%%%%%%%%%%%%%%%%%
%
% sample chapter
%
% Use this file as a template for your own input.
%
%%%%%%%%%%%%%%%%%%%%%%%% Springer-Verlag %%%%%%%%%%%%%%%%%%%%%%%%%%

%\motto{Use the template \emph{chapter.tex} to style the various elements of your chapter content.}
\chapter{An Overview of LTE Advanced}
\label{overview-lte} % Always give a unique label
% use \chaptermark{}
% to alter or adjust the chapter heading in the running head

%\abstract*{Each chapter should be preceded by an abstract (10--15 lines long) that summarizes the content. The abstract will appear \textit{online} at \url{www.SpringerLink.com} and be available with unrestricted access. This allows unregistered users to read the abstract as a teaser for the complete chapter. As a general rule the abstracts will not appear in the printed version of your book unless it is the style of your particular book or that of the series to which your book belongs.
%Please use the 'starred' version of the new Springer \texttt{abstract} command for typesetting the text of the online abstracts (cf. source file of this chapter template \texttt{abstract}) and include them with the source files of your manuscript. Use the plain \texttt{abstract} command if the abstract is also to appear in the printed version of the book.}

\abstract{This chapter provides a high-level overview of LTE-Advanced (LTE-A) networks and associated technologies to form a basis for discussion of the co-existence issues that exist for unlicensed LTE and Wi-Fi. Understanding the underlying architecture and protocols employed in LTE-A networks will provide readers a comparative framework to grasp how, and at what levels, LTE and Wi-Fi networks may interact and interfere with each other, and form a greater understanding of the challenges to be address in designing coexistence mechanisms. Specifically, this chapter will overview the LTE-A network, its capabilities and protocols, with specific emphasis on the physical layer and medium access sub-layers to illuminate specific sources of co-existence issues. Proposed changes which may be included in future LTE releases are discussed in the context of LTE/Wi-Fi coexistence.}


\section{System Overview}
\label{sys-overview}
The enhancements to the Long Term Evolution/System Architecture Evolution (LTE/SAE) to meet the requirements set out for fourth generation (4G) cellular networks are collectively known as LTE-Advanced (LTE-A).  The LTE-A requirements were formalized by the 3rd Generation Partnership Project (3GPP) in LTE releases 10 through 13 \cite{tr36913}.  LTE itself was a logical evolution from the technologies used in previous generations in order to meet the increasing demands for higher data rates and improved quality of service. LTE met these demands at the access level through increased spectral efficiency and improved mobility support and cell edge data rates.  The increased spectral efficiency was achieved by using orthogonal frequency division multiple access (OFDMA) and single-carrier frequency division multiple access (SC-FDMA) in the downlink and uplink, respectively.  Improvements in mobility support and cell edge data rates were achieved through enhanced adaptive modulation and bandwidth selection and downlink spatial multiplexing and multiple input/multiple output support.  Beyond the access layer, LTE transitioned to an all IP packet switched core network with the introduction of the evolved packet core, and a flattened network architecture of enhanced base stations called evolved NodeB's (eNB) interconnected via high-speed data links.  Combined, these fundamental changes to the cellular network architecture has allowed LTE networks to significantly increase user data rates and reduce control and user plane latency and connection set-up and handover times.  LTE-A represents the further, and ongoing, evolution of cellular networks to continue to meet the every increasing demands for higher data rates, user mobility support, and efficient support of a growing number of wireless devices.

\subsection{Network Architecture}
\label{net-arch}
The requirements to provide high data rates while supporting high-speed mobility requires the ability to set up and tear down user connections and manage inter-cell handoffs with as little latency as possible.  In previous generations of cellular networks, a hierarchical structure consisting of base stations or NodeBs connected to a central controller had been used.  This star or cluster architecture requires additional hops in both data transmissions and hand off negotiation which can introduce significant delay.  Controllers were are responsible for managing all data and control traffic, as well as handoffs betweens several pairs of base stations.  For many increasingly ubiquitous end-user applications, such as online gaming and voice/video over the internet, the additional latency in connection set up and handover can impair the user quality of experience.
\begin{figure}[!ht]
	\centering
	\includegraphics[width=0.58\textwidth]{figures3/lteAnet}
	\caption{Basic structure of LTE cellular networks.}
	\label{figs:LTE-A-Network}
\end{figure}

The flat architecture adopted by LTE networks is depicted in Fig. \ref{figs:LTE-A-Network}.  The migration of local functions to eNBs and global functions to the EPC were driven by the requirements of reduced latency and higher data rates. The functions of radio network and medium access control, handoff requests, negotiations, and management, as well as some other truly local functions, are migrated to the eNBs, which are interconnected via high-speed, low latency, connections in a mesh configuration \cite{tr36300}.  The low-latency X2 interface connections between eNBs allows for fast user handover, including forwarding of queued data for seamless user experience.  Additionally, with direct connections between neighboring cells, this architecture facilitates more effective multi-point transmission and coordination and inter-cell interference and load management, independent of conditions in other areas of the network. The global functions and connections to external networks are handled at the evolved packet core (EPC). The functional split between the various components of the network, as well as the implementation of the necessary layers of the network protocol stack, is shown in further detail in Fig. \ref{figs:funcSplit}.
\begin{figure}[!Ht]
	\centering
	\includegraphics[width=\textwidth]{figures3/LTE-decomp}
	\caption{Functional split between various entities in LTE under the system architecture evolution.}
	\label{figs:funcSplit}
\end{figure}
In addition to those listed in the figure, the mobile management entity (MME) handles authentication, authorization and accounting functions.  The packet data network gateway (P-GW) and serving gateway (S-GW) handle user data packet forwarding, filtering, and usage tracking, as well as acting as a mobility anchor for inter-eNB and inter-RAT handovers.  Further, the distributed radio network and resource management and medium access control allows eNBs to quickly adapt to changing radio medium condition and provide timely user scheduling based on local information.

\subsection{Capabilities and Features}

While the gains made by LTE were significant, they fell short of the requirements set out for 4G networks by the International Telecommunications Union, specifically in the case of peak data rates, spectral efficiency, and cell edge performance \cite{itu-advanced}.  The continuing evolution which became LTE-A was finally able to achieve the necessary targets to meet the ITU requirements for 4G.  Some important ITU requirements, and achieved performance levels for LTE and LTE-A, are highlighted in Table \ref{perf-table}.

\begin{table}
	\caption{ITU-A Requirements for 4G vs. LTE/LTE-A Achievements \cite{lte-3gpp}\cite{lteA-3gpp}\cite{itu-advanced}\cite{abdullah}}
	\label{perf-table}      
	\begin{tabular}{p{0.5\textwidth}p{0.15\textwidth}p{0.15\textwidth}p{0.15\textwidth}}
		\hline\noalign{\smallskip}
		Description/Requirements & ITU-A & LTE & LTE-A   \\
		\noalign{\smallskip}\svhline\noalign{\smallskip}
		DL peak spectral efficiency (bps/Hz) &  15   & 15  & 30 \\
		UL peak spectral efficiency (bps/Hz)& 6.75  & 3.75  & 15 \\
		Min. cell edge spectral \\ \hspace{0.8em} efficiency (bps/Hz) & 0.04 & 0.024 & 0.04 \\
		DL Peak data rates (Mbps) & 1000$^1$  & 300 & 1000 \\
		UL Peak data rates (Mbps) & 1000$^1$  & 75 & 500 \\
		Scalable bandwidth up to (MHz) & 40 & 20  & 100$^2$ \\
		
		\noalign{\smallskip}\hline\noalign{\smallskip}
	\end{tabular}
	$^1$ For low mobility with requirement of min. 100 Mbps for speeds of up to 350 km/h. 	 \\
	$^2$ With carrier aggregation of up to five carrier components.
\end{table}

Among other innovations, in order to meet these requirements, LTE-A extended bandwidth scalability in LTE by supporting carrier aggregation, both within and across frequency bands.   Discontiguous aggregation is supported to ensure a higher bandwidth is available for providers who cannot support it in contiguous spectrum allotments, allowing the development of license-assisted access (LAA-LTE) into the unlicensed and TV whitespace bands.  Backwards compatibility is maintained by using bandwidths for each carrier component which match those used in LTE.  LTE-A also expands MIMO/spatial multiplexing support up to 8x8 for DL and 4x4 for UL, adds coordinated multi-point operation to increase spectral efficiency and cell edge data rates, and improves heterogeneous network planning with the enhancement of support for small cells and relay nodes to increase area coverage with reduced power requirements.

\section{Channel Access Mechanisms}
\label{channel-access}
Like other cellular access technologies, LTE-A has been designed for use on dedicated licensed spectrum allocations where there is, generally, no need to contend for channel access.  While interference, fading, and path loss can corrupt LTE transmissions, and recovery and retransmission functions are necessary, in general a centrally controlled and tightly scheduled channel access mechanism is able to guarantee service levels required by all uplink and downlink traffic \cite{tr36300}.  Channel access is achieved through frequency division multiple access (FDMA) and either time or frequency division duplexing.  Orthogonal frequency division (OFDMA) is used in the downlink, allowing the eNB to efficiently schedule transmissions for many users in the same transmission time interval.  Single carrier frequency division (SC-FDMA) is used in the uplink in order to reduce the power consumption requirements of battery dependent user equipment (UE) to communicate with the eNB.  Further, coordinated multipoint (CoMP) is supported by allowing UEs to be configured to process channel state information from multiple eNBs, and both single-user and multi-user MIMO are supported in multiple configurations to achieve transmit diversity or multi-layer transmissions with beamforming possible in both horizontal and vertical dimensions.

The basic structure of the protocol stack used in LTE networks to facilitate channel access is shown in Fig. \ref{figs:stack} \cite{tr36201}.
\begin{figure}[!ht]	
	\includegraphics[width=\textwidth]{figures3/LTEradio-interface}
	\caption{E-UTRA radio interface protocol architecture.}
	\label{figs:stack}
\end{figure}
UL and DL transmissions are divided amongst several physical channels, according to the type of transmission, i.e. user traffic or control information, and the type of transmission, i.e broadcast or unicast and scheduled or random access (random access is primarily used by a UE which has not yet associated to an eNB). The information bearing physical channels are mapped by the PHY layer into transport channels supplied to the MAC sublayer, which in turn remaps these into several logical channels provided to the higher layers. 

\subsection{LTE-A Physical Layer}
\label{lte-phy}
The LTE PHY layer is designed to be both highly adaptable as well spectrally and power efficient.  In the DL, OFDMA is used to schedule many signals in the same transmission time interval and achieve a high spectral efficiency, however, the very high peak to average power ratio makes this multiple access strategy unattractive in the UL for battery dependent devices \cite{tr36201}\cite{tr36211}.  SC-FDMA is used in the UL to maintain a satisfactory spectral efficiency while significantly improving power efficiency and battery life. Example carrier allocations for both DL and UL are shown in Fig. \ref{lte:of-sc-fdma}.

\begin{figure}[!ht] 
	\centering
	\subfloat[OFMDA used in downlink.]{\includegraphics[width=4in]{figures3/ofdma}%
		\label{ofdma}}
	\\
	\subfloat[SC-FDMA used in uplink.]{\includegraphics[width=4in]{figures3/sc-fdma}%
		\label{sc-fdma}}
	\caption{PLACEHOLDER: Multiple access strategies used in LTE for uplink and downlink.}
	\label{lte:of-sc-fdma}
\end{figure}
Each color and shade represents the carriers allocated to a specific user.  As shown in Fig. \ref{ofdma}, LTE-A supports both localized (contiguous) and distributed allocations and the assignment of sub-carriers to a given UE are driven by the specific loss and interference experienced by that user, called channel state information (CSI), in order to maximize the overall achieved rate of all users.  Sub-carrier assignments can further span disjoint frequency bands, including into unlicensed bands, through carrier aggregation.  In Fig. \ref{sc-fdma} a given subcarrier, with variable bandwidth, is assigned to a single UE, again based on CSI.  It should be emphasized that Fig. \ref{lte:of-sc-fdma} shows a single instant of time, over the range of frequency available. 

All transmission are organized into frames comprised of twenty slots, every two of which comprise a subframe \cite{tr36211}. An exampled of one configuration for the LTE frame structure is shown in Fig. \ref{lte:frame}.
\begin{figure}[!t]
	\centering
	\includegraphics[width=0.8\textwidth]{figures3/LTE-frame}
	\caption{LTE frame and resource block structure.}
	\label{lte:frame}
\end{figure}
A single sub-carrier, with either 7.5 or 15 kHz bandwidth, paired with a transmission duration required to transmit a single OFDM symbol is called a resource element. The minimum unit which can be allocated to a physical channel or user is the physical resource block (PRB). Depending on the configuration used, a DRB is formed of either 7, 6, or 3 OFDM symbols and 12 or 24 sub-carriers allocated for one 0.5 ms slot.  Between 6 and 10 PRBs will then be allocated to UE in order to achieve bandwidths between 1.4 and 20 MHz (though the occupied bandwidth will be smaller, 1.08 to 18 MHz).  Higher bandwidths are achieved through carrier aggregation in the same slot, either contiguous or not. 

Two distinct frame structures are defined to support both time division duplexing (TDD) and frequency division duplexing (FDD), as well as a third frame structure specifically for license assisted access (LAA-LTE).  All three frame types have the same basic structure shown in Fig. \ref{lte:frame}, with the differences being in how specific transmission are scheduled.  Both half and full duplex are supported in FDD and 10 subframes are available for both UL and DL.  Transmissions are separated in frequency only, for full duplex, and both time and frequency for half duplex.  For TDD, the frame is organized into two 5 ms half-frames, with several UL/DL configurations and switching patterns supported.  As of Release 13, the special frame structure defined for LAA-LTE reserves all 10 subframes for DL, with transmission able to occupy one or more consecutive subframes, but required to start somewhere within the first subframe.

Beyond the features discussed in detail, the LTE-A PHY layer provides forward error correction and automatic repeat request functions, modulation/demodulation of physical signals, mapping and rate matching of physical channels to transport channels, frequency and time synchronization, and MIMO antenna processing, including transmit diversity and beamforming.  A variety of modulation schemes are supported depending on channel conditions, distance from receiver, and power requirements (up to 256 QAM in the downlink).  

%TODO: continue editing from this point forwards
\subsection{LTE-A Medium Access Control}
\label{lte-mac}
Medium access is tightly controlled and scheduled in both UL and DL in LTE-A, with the eNB controlling the time and frequency resource block assignment for all but random access channels (used for UE connection requests and some other procedures) \cite{tr36321}.  Distinct MAC sublayers for the UE and the eNB are defined, as well as several configurations for UE, depending on the specific functions implemented, such as dual connectivity (similar to carrier aggregation but across small and macro cells) or sidelink for UE to UE communication. The basic structure of the MAC sublayer is shown in Fig. \ref{figs:lte-mac}.

\begin{figure}[!ht]
	\centering
	\includegraphics[width=\textwidth]{figures3/LTE-MAC}	\caption{LTE MAC sublayer structure and channel mapping.}
	\label{figs:lte-mac}
\end{figure}
The MAC sublayer facilitates reliable data transfer through radio resource allocation and the mapping and multiplexing between transport channels and one or more logical channels.  Further, the MAC sublayer handles HARQ signaling, channel prioritization with dynamic scheduling, random access control, data transfer, and scheduling information reporting and requests.  

The specific implementation of these, and other layer 2 functions such as radio link control, are quite complex and beyond the scope of this brief, however the underlying design paradigm is that global of knowledge facilitating tight control over medium usage. The MAC sublayer in the eNB is aware of the channel conditions for each UE to be scheduled in both the UL and DL, and the UE adheres to the schedule of time and frequency provided by the eNB.  PRBs are assigned to UL and DL transmissions to achieve the greatest advantage from local channel conditions. Little consideration of inter-user or inter-carrier interference is necessary, due to the dedicated frequency bands and orthogonal carriers discussed in Sec. \ref{lte-phy}.  

%Consider section change to sources of coexistence issues
\section {Changes Expected for Future Releases}
\label{fut-chnge}
%Am considering dropping this section as it may not be interesting in the context of the rest of the book, would be either rather short OR very technical, and possible a section (such as in my 610 paper) detailing how the preceding sections contribute to potential sources of coexistence issues
LTE-A brings cellular networks into the realm of 4G, as defined by \cite{itu-advanced}.  As far as LTE-A has taken us, it will not be enough for fifth generation networks which are expected to support use cases ranging from smart cities and Internet of Things devices to  self-driving vehicles and industrial automation with \emph{gigabyte} per second, high speed mobile broadband \cite{itu-2020}.  In The ITU has set targets for 5G around peak and average data rates, spectral and energy efficiency, mobility, capacity and density which current wireless networks are incapable of achieving.  

In order to move towards 5G networks, 3GPP has numerous study items underway and planned for future releases, to meet the ITU requirements in \cite{itu-2020}.  These include significant enhancements to inter- and intra-band carrier aggregation and licenses-assisted access to ISM bands, TV white space, and other under-utilized spectrum resources, as well as multi-carrier enhancements and improved CoMP and device to device communications \cite{lteA-gantt}.  Wireless network virtualization and hardware resource sharing, cloud-based radio access networks, and new system architectures are under consideration to support the requirements around energy efficiency and and low-latency communications, among others.  Additionally, the possibility of moving to an entirely new radio and medium access protocols is also under consideration, which would not be hindered by the need to be backwards compatible, and move forward with only the necessity of meeting the ITU 5G requirements.  


%%%%%%%%%%%%%%%%%%%%%%%% referenc.tex %%%%%%%%%%%%%%%%%%%%%%%%%%%%%%
% sample references
% %
% Use this file as a template for your own input.
%
%%%%%%%%%%%%%%%%%%%%%%%% Springer-Verlag %%%%%%%%%%%%%%%%%%%%%%%%%%

\begin{thebibliography}{99.}%
	
	
	
	\bibitem{cisco} Cisco, ``{Cisco visual networking index: Global mobile data traffic forecast up-date 2015-2019},'' Online, whitepaper, February 2016. [Online]. Available: \url{{http://www.cisco.com/c/en/us/solutions/collateral/service-provider/visual-networking-index-vni/mobile-white-paper-c11-520862.html}}
	
	\bibitem{lteu-forum}``{LTE-U technical report: Coexistence study for LTE-U SDLV1.0 (2015-02)},''Online, LTE-U Forum, Tech. Rep., February 2015. [Online]. Available: \url{http://www.lteuforum.org/uploads/3/5/6/8/3568127/lte-u\_forum\_lte-u\_technical\_report\_v1.0.pdf}
	
	\bibitem{3gpp}{3GPP}, ``Feasibility study on licensed-assisted access to unlicensed spectrum,'' \emph{{TR 36.889 v13.0.0}}, July 2015.
	
	
	\bibitem{qual}``{LTE in unlicensed spectrum: Harmonious coexistence with Wi-Fi},'' Online, Qualcomm, Inc, White Paper, June 2014. [Online]. Available: \url{{https://www.qualcomm.com/documents/lte-unlicensed-coexistence-whitepaper}}
	
	\bibitem{zhang} {R. Zhang, M. Wang, et al}, ``Lte-unlicensed: The future of spectrum aggregation for cellular networks,'' \emph{IEEE Wireless Communications}, vol.~22, no.~3, pp. 150--159, June 2015.
	
	\bibitem{timo} {T. Nihtil{\"a}, V. Tykhomyrov, et al}, ``System performance of lte and ieee 802.11 coexisting on a shared frequency band,'' in \emph{Wireless Communications and Networking Conference , 2013 IEEE}, Shanghai, April 2013,pp. 1038--1043.
	
	\bibitem{google} ``{Comments of {G}oogle, {I}nc. In the Matter of the Office of Engineering and Technology and Wireless Telecommunications Bureau seeks Information on Current Trends in LTE-U and LAA Technology},'' before the FCC, June 2015, docket no. 15-105.
	
	\bibitem{80211}``{IEEE Standard for information technology--Telecommunications and information exchange between systems Local and metropolitan area networks--Specific requirements Part 11: Wireless LAN Medium Access Control (MAC) and Physical Layer (PHY) Specifications},'' \emph{{IEEE Std 802.11-2012 (Revision of IEEE Std 802.11-2007)}}, pp. 818--972, March 2012.
	
	\bibitem{tsi} ``{LTE in a Nutshell},'' Online, Telesystem Innovations Inc., White Paper, 2010.
	
	\bibitem{dulaimi} {A. Al-Dulaimi, S. Al-Rubaye, et al}, ``5g communications race: Pursuit of more capacity triggers lte in unlicensed band,'' \emph{IEEE Vehicular Technology Magazine}, vol.~10, no.~1, pp. 43--51, March 2015.
	
	\bibitem{qual2} ``{Comments of {Q}ualcomm, {I}nc. In the Matter of the Office of Engineering and Technology and Wireless Telecommunications Bureau seeks Information on Current Trends in LTE-U and LAA Technology},'' before the FCC, June 2015, docket no. 15-105.
	
	\bibitem{abinader} {F. M. Abinader, E. P. L. Almeida, et al}, ``Enabling the coexistence of lte and wi-fi in unlicensed bands,'' \emph{IEEE Communications Magazine}, vol.~52, no.~11, pp. 54--61, November 2014.
	
	\bibitem{jian} {Y. Jian, {C.-F} Shih, et al}, ``Coexistence of wi-fi and laa-lte: Experimental evaluation, analysis and insights,'' in \emph{Communication Workshop, 2015 IEEE Int. Conf.}, London, June 2015, pp. 2325--2331.
	
	\bibitem{bhorkar} A.~Bhorkar, C.~Ibars, and P.~Zong, ``On the throughput analysis of lte and wifi in unlicensed bands,'' in \emph{Signals, Systems and Computers, 2014 48th Asilomar Conf.}, Pacific Grove, CA, November 2015, pp. 1309--1313.
	
	\bibitem{cablelabs}J.~Padden, ``{Wi-Fi vs. duty cycled LTE: A balancing act},'' Online, CableLabs, White Paper. [Online]. Available: \url{{http://www.cablelabs.com/wi-fi-vs-duty-cycled-lte/}}
	
	\bibitem{cano2} C.~Cano and D.~J. Leith, ``Coexistence of wifi and lte in unlicensed bands: A proportional fair allocation scheme,'' in \emph{Communication Workshop, 2015 IEEE Int. Conf.}, London, June 2015, pp. 2288--2293.
	
	\bibitem{li} Y.~Li, J.~Zheng, and Q.~Li, ``Enhanced listen-before-talk scheme for frequency reuse of licensed-assisted access using lte,'' in \emph{Personal, Indoor, and Mobile Radio Communications, 2015 IEEE 26th Annu. Int. Symp.}, Hong Kong, August 2015, pp. 1918--1923.
	
	\bibitem{kwan} {R. Kwan, R. Pazhyannur, et al}, ``Fair co-existence of licensed assisted access lte (laa-lte) and wi-fi in unlicensed spectrum,'' in \emph{Computer Science and Electronic Engineering Conf., 2015 7th}, Colchester, United Kingdom, September 2015, pp. 13--18.
	
	\bibitem{cano1} C.~Cano and D.~J. Leith, ``{Unlicensed LTE/WiFi coexistence: Is LBT inherently fairer than CSAT?}'' November 2015, {ArXiv pre-print, Computer Science - Networking and Internet Architecture, arXiv:1511.06244}.
	
	\bibitem{yoon} {J. Yoon, S. Yun, et al}, ``{Maximizing Differentiated Throughput in IEEE 802.11e Wireless LANs},'' in \emph{{ Proceedings of the 31st IEEE Conference on Local Computer Networks}}, vol.~14, no.~6, November 2006, pp. 411--417.
	
	\bibitem{chou} C.~T. Chou, K.~G. Shin, and S.~Shankar, ``{Contention-based Airtime Usage Control in Multirate IEEE 802.11 Wireless LANs},'' \emph{{IEEE/ACM Transactions on Networking}}, vol.~14, no.~6, pp. 1179--1192, December 2006.
	
	\bibitem{vu} H.~L. Vu and T.~Sakurai, ``{Collision probability in saturated IEEE 802.11 networks},'' in \emph{{Australian Telecommunication Networks \& Applications Conference}}, September 2006, pp. 1--5.
\end{thebibliography}

