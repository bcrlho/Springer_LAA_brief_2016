%%%%%%%%%%%%%%%%%%%%%%preface.tex%%%%%%%%%%%%%%%%%%%%%%%%%%%%%%%%%%%%%%%%%
% sample preface
%
% Use this file as a template for your own input.
%
%%%%%%%%%%%%%%%%%%%%%%%% Springer %%%%%%%%%%%%%%%%%%%%%%%%%%

\preface

%TODO:Write a preface
In order to enhance the efficiency and reliability of the power grid, diversify energy resources, improve power security, and reduce greenhouse gas emission, many countries have been putting great efforts in designing and constructing their smart grid (SG) infrastructures. Smart grid communications network (SGCN) is one of the key enabling technologies of the SG. However, a successful implementation of an efficient and cost-effective SGCN is a challenging task. This Springer brief gives a comprehensive overview of SGCN by investigating its network architecture, communications standards, and quality-of-service (QoS) requirements. Promising wireless communications technologies that could be used for the implementation of the SGCN are also addressed. In addition, two candidate protocols for the neighbor area network (NAN) segment of the SGCN are investigated and compared in order to identify their strengths, weaknesses, and feasibilities. Especially, a proactive parent switching mechanism to improve the resilience of NANs against smart meter failures is also presented and evaluated. As an attempt to identify possible future research trends, this brief also outlines a number of technical challenges and corresponding work directions in the SGCN.

The target audience of this informative and practical brief is researchers and professionals working in the field of wireless communications and networking. The content is also valuable for advanced-level students interested in architecture design, routing protocol development, and implementation of wireless meshnetworks.

\vspace{\baselineskip}
\begin{flushright}\noindent
Montreal, Canada,\hfill {\it Quang-Dung Ho}\\
September 2016\hfill {\it Daniel Tweed}\\
\hfill {\it Tho Le-Ngoc Tweed}\\
\end{flushright}


