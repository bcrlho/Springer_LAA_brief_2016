%%%%%%%%%%%%%%%%%%%%%%preface.tex%%%%%%%%%%%%%%%%%%%%%%%%%%%%%%%%%%%%%%%%%
% sample preface
%
% Use this file as a template for your own input.
%
%%%%%%%%%%%%%%%%%%%%%%%% Springer %%%%%%%%%%%%%%%%%%%%%%%%%%

\preface

Global mobile traffic is expected to increase nearly tenfold between 2014 and 2020 due to increasing number of mobile-connected devices and the explosion of data-hungry mobile applications. Pushing traffic towards the network capacity quickly deteriorates the Quality of Services (QoSs) perceived by the users. Acquiring additional licensed spectrum to increase the capacity of Radio Access Networks (RANs) is certainly very expensive. Mobile operators are also challenged by the “revenue gap”, i.e., the exponential increase in mobile traffic does not generate sufficient additional revenues required for upgrading their RANs. This circumstance has fostered the interest in cost-effective solutions to increase the capacity of RANs. Long-Term Evolution (LTE) in unlicensed bands (U-LTE) is among promising solutions. However, since U-LTE is a nascent LTE technology, there are still various associated concerns and challenges to be addressed.

This brief first presents a comprehensive survey on U-LTE, focusing on technical issues and the impacts of this technology to other neighboring networks in shared frequency bands. Specifically, concepts, motivations, benefits, obstacles, and coexistence requirements of U-LTE are presented. Three typical types of U-LTE including LTE-U, LAA-LTE, and MuLTEfire are explained. Next, regulations specified by standard institutes for radio systems operating in unlicensed spectrum are reviewed. Third, due to the fact that technical knowledge on medium access mechanisms of LTE and IEEE 802.11/Wi-Fi technologies is strongly required to understand and analyze the interactions between these two technologies when they operate in the same frequency band, high-level network architectures and technical details of LTE and Wi-Fi are presented. Especially, distinguishing features of CSMA/CA employed by Wi-Fi networks compared to standardized regulations are highlighted. Forth, in order to capture the ongoing activities on U-LTE’s coexistence mechanisms, related works are surveyed with insight observations on their limitations and concerns. 

This brief also presents our Network-Aware Adaptive LBT mechanism (NALT) which is proposed for LTE networks for its coexistence with Wi-Fi networks. It a nutshell, the NALT monitors both channel conditions and usage activity to maximize its transmission opportunities, while maintaining fair sharing of the channel, in a way that is transparent to incumbent Wi-Fi devices. Finally, towards future working directions, in the light of the survey, this brief identifies a number of open technical questions as well as related potential research issues in U-LTE.

The findings in this brief provide telecom engineers, researchers, and academic professionals with valuable knowledge and potential working or research directions when designing and developing medium access protocols for next generation wireless access networks.

\vspace{\baselineskip}
\begin{flushright}\noindent
Montreal, Canada,\hfill {\it Quang-Dung Ho}\\
September 2016\hfill {\it Daniel Tweed}\\
\hfill {\it Tho Le-Ngoc}\\
\end{flushright}