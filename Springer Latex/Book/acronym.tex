%%%%%%%%%%%%%%%%%%%%%%acronym.tex%%%%%%%%%%%%%%%%%%%%%%%%%%%%%%%%%%%%%%%%%
% sample list of acronyms
%
% Use this file as a template for your own input.
%
%%%%%%%%%%%%%%%%%%%%%%%% Springer %%%%%%%%%%%%%%%%%%%%%%%%%%
%useful awk: 
%   awk '/^$/ {next}{printf "\item[%s]{%s}\n",substr($1, index($0,$0)), substr($0, index($0,$2))}' < [inputfile] > [outputfile]
\extrachap{Acronyms}
%TODO: Compile/edit list of Acronyms, have text file and awk function which can parse and add latex 
\begin{description}[CABR]
\item[3G,4G,5G]{Third, Fourth, Fifth Generation cellular networks}
\item[3GPP]{3rd Generation Partnership Project}
\item[AC]{Access Category}
\item[ACK]{Acknowledgment, a postive response of reception }
\item[AP]{Access Point, Wi-Fi station which acts as base station in infrastructure mode}
\item[BI]{Backoff Interval}
\item[CAP]{Controlled Access Phase}
\item[CCA]{Clear Channel Assessment}
\item[CFP]{Contention-free Period, in which polling is completed under the PCF}
\item[CoMP]{Coordinated Multi-Point }
\item[CoT]{Channel Occupancy Time}
\item[CP]{Contention Period, in which STAs contend with each other under the PCF}
\item[CSAT]{Carrier-sense Adaptive Transmission, a proposed LAA-LTE/Wi-Fi co-existence mechanism}
\item[CSMA/CA]{Carrier-sense Multiple Access with Collision Avoidance}
\item[CTS]{Clear to Send, channel reservation response in 802.11 Wi-Fi}
\item[CW]{Contention Window, range of possible backoff values a Wi-Fi station will select from}
\item[DCF]{Distributed Coordination Function, a random access based MAC protocol used in IEEE 802.11 Wi-Fi}
\item[DCS]{Dynamic Channel Selection}
\item[DIFS]{DCF Inter-frame Space, minimum delay after channel becomes free before a Wi-FI station will attempt to transmit}
\item[DL]{Downlink, transmission from base station to user equipment}
\item[ECCA]{Extended Clear Channel Assessment, additional CCA done in ETSI LBT when channel is known to have been recently busy }
\item[EDCA]{Enhanced Distributed Channel Access, QoS enhancements for Wi-Fi in 802.11e }
\item[EDCF]{Enhanced Distributed Coordination Function,  a random access based MAC protocol used in IEEE 802.11e Wi-Fi}
\item[EDGE]{Enhanced Data rates for GSM Evolution}
\item[ED]{Energy Detect}
\item[EIRP]{Equivalent Isotropically Radiated Power}
\item[eNB]{eNodeN, an evolved Node B base stations used in LTE/LTE-A networks}
\item[EPC]{Evolved Packet Core}
\item[ETSI]{European Telecommunications Standards Institute}
\item[E-UTRA]{Evolved Universal Terrestrial Radio Access }
\item[E-UTRAN]{Evolved Universal Terrestrial Radio Access Network}
\item[FBE]{Frame-based Equipment}
\item[FCC]{Federal Communications Commission}
\item[FDD]{Frequency Division Duplex}
\item[GC]{Global Controller}
\item[GSM]{Global System for Mobile communication}
\item[HARQ]{Hybrid Automatic Repeat Request }
\item[HCCA]{HCF Controlled Channel Access}
\item[HCF]{Hybrid Coordination Function, a polling based MAC protocol used in IEEE 802.11e Wi-Fi}
\item[HetNet]{Heterogeneous Network, a wireless network made up of different types of access nodes }
\item[HSPA]{High Speed Packet Access}
\item[IEEE]{Institute of Electrical and Electronics Engineers }
\item[IFS]{Inter-frame Space}
\item[IoT]{Internet of Things}
\item[IP]{Internet Protocol}
\item[ITU]{Internation Telecommunications Union}
\item[LAA/LAA-LTE]{Licensed-Assisted Access LTE using LBT}
\item[LBE]{Load-based Equipment}
\item[LBT]{Listen-Before-Talk Medium Access Strategy}
\item[LTE]{Long Term Evolution, 3GPP Releases 8 and 9}
\item[LTE-A]{LTE-A LTE Advanced, 3GPP Releases 10 to 13, possibly more}
\item[LTE-U]{LTE-U LTE in Unlicensed bands using a duty-cycled coexistence mechanism }
\item[MAC]{Medium Access Control, sublayer of the Data Link Layer}
\item[MIMO]{Multiple Input Multiple Output }
\item[MME]{Mobility Management Entity}
\item[MTC]{Machine Type Communications }
\item[NALT]{Network-aware Adative Listen-Before-Talk, a proposed LAA-LTE/Wi-Fi co-existence mechanism}
\item[NAV]{Network Allocation Vector}
\item[OFDM]{Orthogonal Frequency Division Multiplexing }
\item[OFDMA]{Orthogonal Frequency Division Multiple Access}
\item[PCF]{Point Coordination Function, a polling based MAC protocol used in IEEE 802.11 Wi-Fi}
\item[PDN]{Packet Data Network}
\item[P-GW]{PDN Gateway}
\item[PHY]{Physical Layer}
\item[PRB]{Physical Resource Block}
\item[QAM]{Quadrature Amplitude Modulation }
\item[QoS]{Quality of Service}
\item[RAN]{Radio Access Network}
\item[RAT]{Radio Access Technology}
\item[RC]{Regional Controller}
\item[RSSI]{Received Signal Strength Indicator, measure of received power in wireless networks }
\item[RTS]{Request to Send, channel reservation request in 802.11 Wi-Fi}
\item[Rx]{Receiver or Reception}
\item[SC-FDMA]{Single-Carrier Frequency Division Multiple Access }
\item[SDL]{Supplemental Downlink}
\item[SDN]{Software Defined Networking}
\item[S-GW]{Serving Gateway}
\item[SIFS]{Short Inter-frame Space, delay before a response is made or expected to certain frames in 802.11}
\item[ST]{Slot Time, duration of an OFDM symbol in 802.11}
\item[STA]{Station, a member of wireless network }
\item[TC]{Traffic Classes}
\item[TDD]{Time Division Duplex}
\item[TDM]{Time Division Multiplex}
\item[TS]{Traffic Streams}
\item[Tx]{Transmitter or Transmission}
\item[TXOP]{Transmission Opportunity}
\item[UE]{User Equipment}
\item[UL]{Uplink}
\item[U-LTE]{LTE LTE in Unlicensed bands, catchall phrase to cover all methods of using LTE in unlicensed frequency bands}
\item[UMTS]{Univesral Mobile Telecommunications System}
\item[VoIP]{Voice over IP}
\item[VoLTE]{Voice over LTE}
\item[WLAN]{Wireless Local Area Network}
\end{description}